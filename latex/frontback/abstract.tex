\chapter*{Abstract}
	\doublespacing

	Deep tissue injuries---casually termed ``bedsores''---are a type of wound often suffered as secondary injuries in persons with reduced mobility such as those with spinal cord injuries, the elderly, and people with multiple sclerosis. Deep tissue injuries form when excess pressure is applied to tissue for extended periods of time without relief and begin as the necrotic breakdown of deep tissues at the bone-muscle interface. As these wounds progress, they ``tunnel'' up toward the surface of the skin where they break open into late-stage pressure ulcers and are completely unnoticeable if the patient lacks sensation. There is currently no clinical tool for detecting these injuries as they form and progress and as such these wounds represent a severe burden on both patients and the health care system alike.

	The goal of this work was to numerically characterize the use of three modalities of ultrasound elastography---quasi-static ultrasound elastography, acoustic radiation force impulse imaging, and shear wave speed quantification---for the early detection of deep tissue injuries. Ultrasound elastography is an imaging modality capable of imaging the mechanical stiffness of soft tissue which is a key measure of tissue health. Through combinations of k-space pseudo-spectral techniques, finite-element modelling, and image processing methods, the ability of ultrasound elastography to detect and accurately diagnose deep tissue injuries was explored. Parametric studies were undertaken to investigate the effect of a wide range of parameters relevant to the detection sensitivity of ultrasound elastography including both device-design parameters and deep tissue injury lesion properties.

	Through the numerical characterizations performed in this work, an understanding of the benefits and limitations of using ultrasound elastography to detect deep tissue injuries was achieved. Shear wave speed quantification was found to provide the most accurate measures of tissue health, however its effectiveness may be limited in very deep tissues. The understanding gained from this work may lead to investigating ultrasound elastography as a viable detection modality for early deep tissue injuries in both animal models and human subjects---these real world tests are the next step on the way to clinical adoption of ultrasound elastography for the early detection of deep tissue injuries.