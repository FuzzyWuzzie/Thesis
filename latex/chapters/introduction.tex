\chapter{Introduction}
	\section{Motivation}
		Pressure ulcers are debilitating wounds often suffered by people with limited mobility such as those undergoing lengthy surgical procedures, the elderly, and those with spinal cord injuries (SCI) \cite{allman95}---up to \SI{80}{\percent} of people with SCI will develop a pressure ulcer in their lifetime \cite{salzberg96}. Pressure ulcers are generally characterized by a deterioration of the skin leading to painful open wounds and while many pressure ulcers may be blamed on excess friction and moisture at the skin surface, many can start as ``deep tissue injuries'' (DTI) which start deep below the skin surface---most often at the bone-muscle interface \cite{kanno09}. DTI are generally thought to be formed due to some combination of excessive deformation and ischemia resulting from sustained loading on localized tissue \cite{stekelenburg08, gefen05, loerakker11}. As of the time of writing, there is no clinically feasible method of detecting deep tissue injuries until they begin to damage the skin---even the National Pressure Ulcer Advisory Panel's description of them is largely based on their appearance after the fact \cite{npuap07}. With our inability to detect these forming injuries and subsequently implement deep tissue injury prevention and mitigation protocols, the injuries may eventually progress to form large subcutaneous cavities which eventually break through the surface and reveal themselves as stage III or IV pressure ulcers \cite{bouten03,oomens10}.

		Currently, the only tool capable of readily detecting early deep tissue injuries is $T_2$-weighted MRI \cite{stekelenburg06,loerakker11}. Unfortunately, MRI is not cost-effective for detecting the onset of DTI in a clinical population. Alternately, ultrasound is a much more cost-effective, if less sensitive imaging modality. While it has been shown that some DTI may be discerned using classical b-mode ultrasound imaging \cite{aoi08, kanno09}, the sonographic features of DTI are difficult to separate from regular tissue inhomogeneities. To overcome this, ultrasound elastography may provide more reliable results by imaging the mechanical tissue stiffness rather than its acoustic properties. Ultrasound elastography is an imaging modality which utilizes sonographic techniques to determine the localized mechanical stiffness of tissue and is currently used clinically to detect breast and prostate cancer lesions \cite{tanter08, konig05} as well as liver fibrosis \cite{sandrin03}. It is known that as DTI form, they undergo mechanical stiffness changes throughout their progression \cite{linderganz04,oomens10,solis12-03}, with tissue undergoing significant 1.8 -- 3.3-fold mechanical stiffening during injury formation \cite{gefen05}. Initially damaged tissues show signs of increased relative stiffness due to edema-related swelling, while eventually showing signs of decreased relative stiffness due to decomposition and necrosis \cite{gefen09}. Since ultrasound elastography is capable of imaging these stiffness changes, it follows that the formation and progression of DTI may be imaged using ultrasound elastography. In fact, ultrasound elastography has shown to be a valid technique for imaging the formation of a DTI in a rat model \cite{deprez11}. Before this technique can be fully understood and used in human patients, the various parameters involved in performing ultrasound elastography must be characterized with respect to detecting DTI in humans.

	\section{Objective}
		The broad objective of this work was to numerically characterize the use of ultrasound elastography to detect and monitor formative and progressive deep tissue injuries. Although it has been shown that ultrasound elastography is capable of imaging DTI \cite{deprez11}, the degree of suitability of this technique with regard to DTI is not yet understood. When the effect of numerous interrogation parameters on detection sensitivity and ability is known, the technology may be evaluated on its feasibility and usefulness to detect deep tissue injuries. The ultimate goal of this characterization is to be the first stage in the process of allowing ultrasound elastography to be implemented clinically for detecting DTI. It is reasoned that if early detection modalities are implemented clinically, both patients and the health care system may benefit by lowering the incidence and outright cost of treating fully-formed deep tissue injuries.

	\section{Methodology}
		In order to investigate the use of ultrasound elastography for the detection of deep tissue injuries, the technology must first be characterized and fully understood. While traditional experimentation provides an opportunity to work with physical subjects it can be severely limiting as absolute control over all investigated parameters is relinquished. Further, subject recruitment may present an insurmountable barrier to the execution of such a study. As such, in this exploratory work, various numerical models of the technology have been utilized to investigate the controlled effect of a broad number of parameters relating to each technology. Specifically, k-space models of ultrasonic wave propagation and finite-element models of tissue deformation have been developed. These models were coupled with tissue strain estimation algorithms to fully simulate the ultrasound ultrasound elastography procedures. Parametric studies on the detection sensitivity and ability of the various ultrasound elastography procedures were carried out with respect to various lesion and technological parameters. Chief parameters of interest included those related to the physical realities of deep tissue injuries such as lesion depth, size, and relative mechanical stiffness as well as parameters related to the design and development of appropriate ultrasonic transducers such as probing frequency, transducer f-number, etc.

	\section{Thesis Outline}
		In this work, three methods of ultrasonic elastogram image formation have been investigated: quasi-static ultrasound elastography, acoustic radiation force impulse imaging, and shear wave speed quantification. While all three methods may be used to interrogate tissue stiffness, each does so in a distinctively unique way. The academic background leading to the motivation for this work and the development of the numerical models is presented in Chapter \ref{chap:litreview}.

		Quasi-static ultrasound elastography estimates tissue strain by tracking inhomogeneities across pre- and post- compression b-mode scans where the compression is generated by manual indentation of the transducer against the surface of the skin. Naturally, mechanically stiffer regions of tissue will strain significantly less than the relatively unstiff surrounding tissue. To investigate this technique, two-dimensional b-mode ultrasound scans of simulated pre- and post-compression tissue were generated. A finite-element model of tissue deformation was utilized to generate the post-compression simulated scans. A published tissue strain estimation algorithm was utilized to then generate elastograms for the parametric study. The models and results pertaining to this technique are presented in Chapter \ref{chap:quasi-static}.

		Acoustic radiation force impulse imaging estimates tissue strain by applying an acoustic radiation body load to deep tissue, causing the interrogated tissue region to deform which is then tracked using conventional ultrasound beams fired at a high frame rate. The magnitude of deformation and tissue relaxation time may then be correlated to tissue stiffness. To simulate this procedure, a k-space pseudospectral method was used to generate simulated acoustic body loads which were then combined with a finite-element model of tissue deformation to analyze the sensitivity of ARFI imaging to investigate formative DTI. The models and results for this technique are presented in Chapter \ref{chap:arfi}.

		Shear wave speed quantification quantifies tissue stiffness by tracking shear waves generated by an acoustic radiation force impulse and correlating the speed of the relatively slow-moving generated shear waves to the mechanical stiffness of the tissue. To simulate shear wave speed quantification, the k-space pseudospectral method of simulating acoustic body loads adopted in Chapter \ref{chap:arfi} was used in combination with a finite-element model of tissue deformation to investigate the interaction between lesions and shear wave speed. The models and results for this technique are presented in Chapter \ref{chap:shear}.

		Finally, the conclusions derived from this work and their implications along with suggestions for future studies are discussed in Chapter \ref{chap:conclusion}.

	\comment{Pressure ulcers are debilitating wounds often suffered by people with limited mobility such as those undergoing lengthy surgical procedures, the elderly, and those with spinal cord injuries (SCI). Pressure ulcers are generally characterized by a deterioration of the skin leading to painful open wounds and while many pressure ulcers may be blamed on excess friction and moisture at the skin surface, many can start as ``deep tissue injuries'' (DTI) which start deep below the skin surface---often at the bone-muscle interface. Due to their ``deep'' nature, DTI are not readily detectable at the surface of the skin and there is currently no clinically feasible method of detecting them. Once a DTI sets in, it will often progress to form large cavities under the skin which eventually break through the surface and reveal themselves as stage III or IV pressure ulcers.

	Ultrasound elastography is a relatively new ultrasonic imaging modality which utilizes traditional ultrasound waveforms to interrogate tissue stiffness rather than tissue echogenicity as is done in classic ultrasound imaging. The resulting tissue stiffness maps are referred to as elastograms, a terminology which will be used throughout this work. By examining displacement characteristics of tissue under load, the relative localized stiffness of the tissue may be ascertained. While regional tissue stiffness changes are to be expected due to the heterogeneous composition of generalized soft tissues, localized stiffness changes may be used as an indicator of tissue health \cite{gefen09} with relatively stiff tissues showing signs of rigor mortis and cell death and relatively soft tissues showing signs of tissue necrosis and decomposition. While ultrasound elastography has typically been used to investigate cancerous lesions this work seeks to use it as a means of detecting deep tissue injuries which as of the time of writing are not clinically detectable until they breach the surface of the skin.

	Before ultrasound elastography can be used clinically with any degree of certainty, the effect of numerous important parameters relating to the imaging modality must be understood and characterized. For example, chief parameters of interest include the depth of a lesion and its overall size---parameters which may immediately disqualify certain lesions from even being interrogated by diffused ultrasound beams and as a result would be invisible on the subsequent elastogram. Similarly, for the purposes of designing application-specific elastography transducers it becomes critical to fully understand the effect of transducer device parameters such as ultrasonic probing frequency and transducer f-number on the elastogram icanamage quality and lesion contrast. In order to properly use ultrasound elastography to detect, diagnose, and monitor formative and progressive deep tissue injuries it is crucial to first fully understand and characterize the technology for this specific use.

	\section{Objective}
		The broad objective of this work was to numerically characterize the use of ultrasound elastography to detect and monitor formative and progressive deep tissue injuries. When the effect of numerous interrogation parameters is understood, the technology may be evaluated on its feasibility and usefulness to detect deep tissue injuries, with the ultimate goal that ultrasound elastography be implemented clinically for detecting deep tissue injuries.

	\section{Motivation}
		According to the National Pressure Ulcer Advisory Board, deep tissue injuries are classified as a sub-category of pressure ulcers \cite{black11}. Pressure ulcers and subsequently deep tissue injuries are commonly suffered by people with limited mobility, such as those undergoing lengthy surgical procedures, the elderly, and those with spinal cord injuries \cite{allman95} with up to \SI{80}{\percent} of people with spinal cord injuries developing at least one pressure ulcer in their lifetime \cite{salzberg96}. While traditional pressure ulcers form in a ``top-to-bottom'' pattern [??], deep tissue injuries form in a ``bottom-to-top'' pattern, whereby the injury starts deep below the skin surface---often at the bone-muscle interface \cite{kanno09}. This nature of not being externally visible until the wound has severely progressed makes deep tissue injuries exceedingly difficult to not only diagnose but also to prevent and treat.

		As of the time of writing, there is no clinically feasible method of detecting deep tissue injuries until they begin to damage the skin---even the National Pressure Ulcer Advisory Panel's description of them is largely based on their appearance after the fact \cite{npuap07}. With our inability to detect these forming injuries and subsequently implement deep tissue injury prevention and mitigation protocols, the injuries may eventually progress to form large subcutaneous cavities which eventually break through the surface and reveal themselves as stage III or IV pressure ulcers \cite{bouten03,oomens10}. \note[KH]{Needs more}

	\section{Methodology}
		In order to investigate the use of ultrasound elastography for the detection of deep tissue injuries, the technology must first be characterized and fully understood. While traditional experimentation provides an opportunity to work with physical subjects it can be severely limiting as absolute control over all investigated parameters is relinquished. Further, subject recruitment may present an insurmountable barrier to the execution of such a study. As such, in this exploratory work, various numerical models of the technology have been utilized to investigate the controlled effect of a broad number of parameters relating to each technology. Specifically, finite-element models of ultrasonic wave propagation in heterogeneous soft tissues have been developed. \annote[KH]{These models were coupled with various tissue strain estimation algorithms and utilized to carry out parametric studies on the detection sensitivity of ultrasound elastography with respect to various lesion and technological parameters.}{Holy wordy batman!} Chief parameters of interest include those related to the physical realities of deep tissue injuries such as lesion depth, size, and relative mechanical stiffness as well as parameters related to the design and development of appropriate ultrasonic transducers such as probing frequency, transducer f-number, etc.

	\section{Thesis Outline}
		In this work, three methods of ultrasonic elastogram image formation have been investigated: quasi-static ultrasound elastography, acoustic radiation force impulse imaging, and shear wave speed quantification. While all three methods may be used to interrogate tissue stiffness utilizing the principles of ultrasound physics.}

\comment{
	\cleardoublepage

	\phantomsection

	\addcontentsline{toc}{section}{References}
	\bibcomplete{references}
	\printbibliography[heading=subbibliography]
}