\chapter{Literature Review}
\label{chap:litreview}
	In order to understand the need for a clinical method of detecting deep tissue injuries, the full scope of the issue must be explored. To this end, the current state of the literature regarding deep tissue injuries, how they form, what factors characterize them, and how they are currently treated is explored here. In order to relate this disease to the detection modalities proposed in this work, the mechanics and history of ultrasound elastography are also explored and related back to the problem at hand. The major gaps in the current literature regarding the use of ultrasound elastography for detecting and monitoring deep tissue injuries are presented as this work attempts to partially fill those gaps and bring the technology one step closer to clinical implementation.

	\section{Deep Tissue Injuries}
		Pressure ulcers, commonly referred to as ``bedsores'', are an extraordinarily large problem facing the health care system today. At least \SI{11}[\$]{billion} is spent in the United States of America alone treating approximately 500,000 injuries annually \cite{beckrich99,russo08} while only a minute fraction of that is spent toward pressure ulcer research \cite{zanca03}. Compared to hospital stays for all other conditions, patients with at least a secondary diagnosis of a pressure ulcer were more often discharged to a long-term care facility and more likely resulted in death \cite{russo08}. These injuries place an extremely significant burden on the people who suffer from them---pressure ulcers were found to have a profound impact on people's lives including: altering their physical, social, and financial status; changing their body image; losing independence and control; and subjecting them to the grieving process \cite{langemo00,baharestani94}. These debilitating wounds are often suffered by people with limited mobility such as those undergoing lengthy surgical procedures, the elderly, and those with spinal cord injuries (SCI) \cite{allman95}---approximately 80 percent of people with spinal cord injuries (SCI) will develop at least one pressure ulcer during their lifetime \cite{salzberg96} and approximately 19 percent of elderly patients in long-term care facilities will develop one \cite{freitas11}. Pressure ulcers exist throughout the entire health-care system and are often formed when undergoing hospitalization \cite{aronovitch99}. These injuries have a tendency to become chronic, non-healing wounds and many patients die from complications related to them \cite{jaul10}. Furthermore, patients who have developed at least one pressure ulcer in their life are at a significantly greater risk of developing a second one \cite{niazi97}. 

		Pressure ulcers generally form over boney prominences with approximately \SI{64}{\percent} occurring over the ischial tuberosities, trochanter, or sacrum \cite{garber03} and typically start at the surface of the skin and progress deep in the tissue. Deep tissue injuries---currently defined as a type of pressure ulcer---form in the same regions are pressure ulcers but generally form at the bone-muscle interface deep in the tissue \cite{kanno09}. In general, these injuries are characterized by a some manner of tissue loss through necrosis of the tissue, though there is currently some debate on the exact nature of these wounds as well as the accuracy of the clinical descriptions attributed to them by the National Pressure Ulcer Advisory Panel (NPUAP).

		The NPUAP defines pressure ulcers as a ``localized injury to the skin and / or underlying tissue usually over a bony prominence, as a result of pressure, or pressure in combination with shear and / or friction'' and are generally staged according to a tiered system of increasing damage \cite{npuap07}. The various stages of pressure ulcer classifications are depicted in Fig. \ref{fig:npuap-staging} and described as follows \cite{npuap07}:

		\begin{figure*}[!t]
			\centering
			\subfloat[Normal tissue]{\includegraphics[width=0.3\textwidth]{assets/npuap/normal.png}\label{fig:npuap-normal}}

			\subfloat[Stage I]{\includegraphics[width=0.3\textwidth]{assets/npuap/stage1.png}\label{fig:npuap-stage1}}
			\subfloat[Stage II]{\includegraphics[width=0.3\textwidth]{assets/npuap/stage2.png}\label{fig:npuap-stage2}}
			\subfloat[Stage III]{\includegraphics[width=0.3\textwidth]{assets/npuap/stage3.png}\label{fig:npuap-stage3}}

			\subfloat[Stage IV]{\includegraphics[width=0.3\textwidth]{assets/npuap/stage4.png}\label{fig:npuap-stage4}}
			\subfloat[Unstageable]{\includegraphics[width=0.3\textwidth]{assets/npuap/unstageable.png}\label{fig:npuap-unstageable}}
			\subfloat[Suspected DTI]{\includegraphics[width=0.3\textwidth]{assets/npuap/suspectedDTI.png}\label{fig:npuap-dti}}
			\caption[NPUAP pressure ulcer staging guidelines]{The NPUAP staging guideline illustrations of the various stages / severities of pressure ulcers.}
			\label{fig:npuap-staging}
		\end{figure*}

		\begin{description}
			\item[Suspected Deep Tissue Injury] \hfill \\
				Purple or maroon localized area of discoloured intact skin or blood-filled blister due to damage of underlying soft tissue from pressure and / or shear. The area may be preceded by tissue that is painful, firm, mushy, boggy, warmer or cooler as compared to adjacent tissue.
			\item[Stage I] \hfill \\
				Intact skin with non-blanchable redness of a localized area usually over a bony prominence. Darkly pigmented skin may not have visible blanching; its colour may differ from the surrounding area.
			\item[Stage II] \hfill \\
				Partial thickness loss of dermis presenting as a shallow open ulcer with a red pink wound bed, without slough. May also present as an intact or open / ruptured serum-filled blister.
			\item[Stage III] \hfill \\
				Full thickness tissue loss. Subcutaneous fat may be visible but bone, tendon or muscle are not exposed. Slough may be present but does not obscure the depth of tissue loss. \emph{May} include undermining and tunnelling.
			\item[Stage IV] \hfill \\
				Full thickness tissue loss with exposed bone, tendon or muscle. Slough or eschar may be present on some parts of the wound bed. Often include undermining and tunnelling.
			\item[Unstageable] \hfill \\
				Full thickness tissue loss in which the base of the ulcer is covered by slough (yellow, tan, grey, green, or brown) and / or eschar (tan, brown or black) in the wound bed.
		\end{description}

		The NPUAP's definitions of pressure ulcers arise from clinical experiences with them and are largely based on the ulcer's appearance after they have formed and do not necessarily reflect the true aetiological factors that lead to these conditions. For example, a significant body of literature scientifically describes deep tissue injuries as being much more insidious than a ``localized area of discoloured intact skin'' and suggests that many Stage III and IV pressure ulcers are actually advanced deep tissue injuries rather than advanced Stage I or II ulcers \cite{gefen09}. This chasm between the clinically accepted and scientifically observed definitions of deep tissue injuries is likely due to the lack of any clinical detection ability \cite{campbell10}. What is agreed upon is that deep tissue injuries are a major problem and more needs to be done to facilitate preventing and treating them \cite{black11,maklebust05}. One of the largest hurdles to preventing and treating DTI is the lack of any substantial early detection ability \cite{gunningberg08,milne09}.

		\subsection{Aetiology and Histology}
			\label{sec:litreview-aetiology}
			Deep tissue injuries are thought to occur through the combinatory effects of three distinct but related mechanisms: ischemia, insufficient lymph drainage, and cell deformation. Ischemia is a condition where the blood supply to tissue has been cut off, rendering the tissue unable to function appropriately. Insufficient lymph drainage refers to how waste products may accumulate in tissue when the lymph vessels that normally carry them away become occluded. Cell deformation occurs when mechanical strains are imparted upon the tissue, causing excessive deformation in not only the extracellular matrix, but in the cells as well. Taken together, the presence of these factors has been shown to greatly increase the risk of developing a deep tissue injury \cite{stekelenburg08}.

			For quite some time, ischemia was regarded as the chief acute risk factor for developing late-stage pressure ulcers \cite{witkowski82,dinsdale74,kosiak61}. Although studies have shown that healthy tissue is able to survive complete ischemia for approximately 4 hours before severe necrosis sets in \cite{labbe87,strock69}, deep tissue injuries are clinically found when loading times are substantially less than this \cite{aronovitch99,bliss99}. The model of ischemic damage alone could not account for the rate of late-stage pressure ulcers that we were witnessed.

			Once it was realized that ischemia alone could not be the culprit behind deep tissue injury formation, ischemia-induced reperfusion injury became implicated in the formation of DTI \cite{Ytrehus95,Blaisdell02,tsuji05}. An ischemia-induced reperfusion injury is caused when blood is allowed to flow back into a region of tissue that was previously ischemic. While seeming somewhat contrary to its expected effect, the restoration of circulation results in a swelling and inflammatory effect which causes extensive microvascular damage \cite{Blaisdell02}. The effect of reperfusion was confirmed when comparing pure ischemic conditions in tissue to a cycle of ischemic-reperfused conditions over the same period of time, where it was found that significantly greater damage was caused by repeated loading and unloading rather than simple constant loading \cite{tsuji05,salcido94}. While ischemia-reperfusion injuries provide a more complete explanation about the formation of deep tissue injuries, they still do not account for those injuries acquired under constant pressure over short time periods.

			In order for cells to function in a healthy manner, the waste they produce must be constantly carried off and processed via the lymphatic system and its series of lymph vessels that perfuse tissue. If the magnitude of pressure applied to tissue reaches a threshold level, the pressure occludes the lymph vessels and lymphatic drainage ceases \cite{miller81}. Once lymphatic drainage ceases, cell waste accumulates in the tissue and is thought to initiate necrosis in the cells \cite{krouskop78,reddy81,braden87}.

			In order to account for deep tissue injuries that form over short time periods, a model of cell deformation leading to necrosis has more recently been proposed \cite{landsman95,bouten01,wang05}. It has constantly been observed that tissue regions which eventually form deep tissue injuries exhibit signs of locally increased strains \cite{stekelenburg06,ceelen08,linderganz08,portnoy09,solis12-03}, with greater degrees of deformation correlating to greater degrees of damage. To account for these results, it has been proposed that excessively deforming strains applied to cells over extended periods of time can alter the permeability of the cell's plasma membranes, leading to an overall reduced cell viability \cite{slomka12}. Further, it has been shown both in finite-element models and experimentally that the stiffness of soft tissue and the corresponding strains that are developed within them are closely related \cite{loerakker13,gefen05,linderganz09,nagel09}. Not only does the amount of deformation depend on the stiffness of tissue, but the stiffness of tissue was found to correlate to the level of deep tissue injury damage seen in the resulting histology \cite{gefen04} with immediate 1.6-fold to 3.3-fold stiffening of the tissue occurring immediately after injury \cite{gefen05,linderganz04}. Further, the stiffness of tissue severely drops below that of healthy tissue when it begins to decompose \cite{gefen05,dimaio01}, leading to a relationship between injury progression and stiffness as shown in Fig. \ref{fig:stiffness-time-relation} (adapted from \cite{gefen09}).

			\begin{figure}[!t]
				\centering
					\begin{tikzpicture}[x=0.75\textwidth,y=0.375\textwidth]
						% basal stiffness
						\draw[ultra thick, dashed]
						(0, 0.3) -- (1, 0.3);

						% stiffness curve
						\draw[ultra thick, draw=pc1] plot[smooth, tension=1] (0, 0.3) .. controls(0.15, 0.3) and (0.15, 1) .. (0.3, 1);
						\draw[ultra thick, ->, draw=pc1] plot[smooth, tension=1] (0.3, 1) .. controls(0.45, 1) and (0.45, 0.1) .. (1, 0.1);

						% axes
						\draw[ultra thick, <->, draw=black] (0, 1) -- (0, 0) -- (1, 0);

						% time tick
						\draw (0.3, -0.05) -- (0.3, 0.05);

						\node[below] at (0.15, 0) {Hours};
						\node[below] at (0.6, 0) {Days};
						\node[rotate=90] at (-0.05, 0.75) {Stiffness};
						\node[left] at (0, 0.3) {\scriptsize Basal Stiffness};
						\node at (0.3, 0.6) {\footnotesize Local Rigor Mortis};
						\node at (0.9, 0.2) {\footnotesize Decomposition};
						\node at (0.9, -0.05) {Time};
					\end{tikzpicture}
				\caption[Schematic representation of the time course of tissue stiffness changes in a deep tissue injury]{Schematic representation of the time course of tissue stiffness changes in a deep tissue injury site. The estimate for the time-course for local rigor mortis was obtained from animal model studies \cite{portnoy08} and the estimate for the time-course for tissue decomposition was obtained from the forensic literature \cite{dimaio01}. (Adapted from Gefen 2009 \cite{gefen09})}
				\label{fig:stiffness-time-relation}
			\end{figure}

			There have been many models of deep tissue injury formation throughout the years, each relating to different mechanisms, though all relating to mechanical stress of the tissue, either through vessel occlusion or direct cellular strain. The truth is most likely a combination of these effects, with cell deformation dominating the damage on shorter time scales with increased applied pressure and vessel occlusion type injuries dominating on longer time scales \cite{stekelenburg08}. In order to further investigate the etiology of PU and DTI, a combination of experimental and numerical studies has been suggested to provide better fundamental knowledge besides existing clinical experience \cite{bouten03}. There is also significant evidence in the literature that suggests that the current NPUAP definitions of PU and DTI are insufficient and not based on scientific evidence and that updating the clinical definitions to better reflect what exists in the literature is crucial to increasing the success of diagnosis and treatment of PU and DTI \cite{gefen09,campbell10}.

			\comment{
				Other possible papers to cite:
					\cite{ceelen08-8}: Computational model that shows how cells that die under compression decrease in stiffness.
					\cite{gefen07-9}: Review of knowledge of DTI aetiology, and why the NPUAP definition is shitty
					\cite{linderganz06}: Greatest strain occurs deep in the tissue, not at the surface
					\cite{then09}: Material information for examining soft tissue deformation
					\cite{vanNierop10}: Diffusion of water affected only by tissue temperature, not deformation
					\cite{salcido94}: Lesions occur deep in tissue rather than at the surface
					\cite{gefen07}: Sitting posture greatly changes the damage that occurs in PU
					\cite{oomens10}: Interface pressure is not a good measure of tissue health, but rather internal strains are
			}

		\subsection{Prevention and Treatment}
			The current state of deep tissue injury treatment and prevention largely reflects the lack of a quantifiable detection modality. One of the most commonly used preventions is called ``turning'' whereby patients are repositioned in their beds or wheelchairs such that individual regions of tissue are intermittently relieved of pressure. Although commonly implemented in health care settings, turning has repeatedly been found to be inadequate at reducing the incidence of pressure ulcers \cite{vanderwee07,rich11}. A more technological means of reducing the mechanical loads on tissue lies in support surface design \cite{krouskop86}. Unlike turning, pressure-redistribution foam mattresses have repeatedly shown their ability to reduce the incidence of pressure ulcers in a cost-effective manner \cite{pham11,rafter11}. Despite the effectiveness of these surfaces, the overall prevalence of pressure ulcers has not changed significantly---suggesting that appropriate preventions are not being utilized in health-care settings \cite{maklebust05}.

			An emerging technology in the realm of pressure ulcer prevention is intermittent electrical stimulation (IES). IES is the process by which electrical impulses are utilized to activate muscle fibres and contract the muscle. IES has been found to not only increase the oxygenation in deep tissue \cite{gyawali11}, but also significantly reduce the damage caused from excessive loading \cite{solis13}. IES prevention paradigms are still being developed but the technology may prove to be an extremely effective preventative therapy for DTI.

			While various technologies exist or are in development for preventing pressure ulcers, little is available to treat them when they occur. Generally, pressure ulcer treatment involves optimizing regional blood flow, managing underlying illnesses, and providing adequate nutrition \cite{jaul10}. If a pressure ulcer has become chronic, treatment switches to controlling the symptoms and preventing complications \cite{jaul10}. Negative pressure wound therapy is a process by which a slight vacuum is applied to the open wound for several weeks and has shown some success in reducing the severity of late-stage pressure ulcers \cite{greer13}. Surgical techniques such as debriding may also be used in an attempt to remove necrotic tissue from the wound and prevent it from growing any larger \cite{longe86,brem02}. Skin-flap surgery is often used on chronic ulcers in an attempt to protect the wound bed \cite{biglari14}.

			When various prevention and treatment paradigms are implemented, the incidence of hospital-acquired pressure ulcers may decrease dramatically \cite{bales11,thompson11,carson11}. However, one of the key required areas of improvement is in the detection and monitoring of pressure ulcers \cite{milne09}---without the ability to continually monitor a wound, the true effectiveness of any given therapy is ultimately indeterminate.

		\subsection{Detection}
			As previously mentioned, there is a lack of means for detecting the early onset of deep tissue injuries in a clinical setting \cite{gunningberg08,milne09}. Currently, when attempting to detect and diagnose a deep tissue injury or pressure ulcer, clinicians generally rely upon a risk-factor scale for patients rather than actually detecting a lesion. Popular risk assessment tools include the Norton, Braden, and Risk Assessment Pressure Sore scales which each attempt to predict the formation of a pressure ulcer in a patient given their scores in a series of relatively subjective variables such as ``general physical condition'' and ``mental state'' \cite{norton63,braden94,lindgren02}. Aside from these main risk-assessment scales, multiple other scales have been proposed for specific populations such as SCI patients \cite{salzberg96} and oncology patients \cite{fromantin11}. While these tools assist health-care practitioners to manage their limited resources with regards to patient care, at best they only provide guesses as to who will develop pressure ulcers or not. The sensitivity---the ability to correctly diagnose an existing condition---of these techniques ranges from approximately \SI{42}{\percent} -- \SI{87}{\percent} while the specificity---the ability to correctly determine that no condition is present---ranges from \SI{57}{\percent} -- \SI{88}{\percent} \cite{kallman14}. Other studies have shown that nurses have great difficulty detecting and diagnosing suspected deep tissue injuries given the current frameworks they are provided \cite{lee13}, while physicians may be even worse \cite{levine12}. While these scales are ``better than nothing'' at diagnosing patients with pressure ulcers, they are far from ideal and are simply not capable of actually diagnosing this disease---for that, a quantifiable detection technology is required.

			In pressure ulcer research it is common to evaluate the extent of deep tissue injury formation through the use of $\mathrm{T}_2^*$-weighted MRI \cite{loerakker11,stekelenburg06,solis12-03}. $\mathrm{T}_2^*$-weighted MRI is able to detect deep tissue injury by investigating tissue oxygenation as a proxy for detecting the lack of cellular activity due to necrosis. Although this technique is well suited for research purposes, it is simply not viable for detecting and monitoring the progression of DTI in the large population of at-risk patients. At the time of writing, MRI scans can easily cost thousands of dollars and take over an hour to complete \cite{wardlaw14,schulthess14,johnson14}. Further, a large proportion of the at-risk population cannot undergo MRI scans for various reasons such as having medical implants or being unable to relocate from their hospital beds to a stationary MRI machine. Of the alternative diagnostic imaging modalities that currently exist, ultrasound provides the most promise due to it's ability to noninvasively interrogate tissues in a mobile and cost-effective manner.

			B-mode ultrasound scans involve the sonographic interrogation of a tissue's acoustic properties by transmitting sound waves on the order of multiple \si{MHz} and ``listening'' to the waves as they are reflected in tissue. B-mode ultrasound imaging has been used to identify hypo-echoic regions in sub-epidermal tissue related to DTI \cite{andersen08,aoi08,kanno09}, however the results from these studies are somewhat unclear and require a degree of interpretation of the results. After combining thermographic techniques with b-mode imaging, it may be possible to increase the accuracy of early deep tissue injury detection \cite{higashino12}. As a more reliable alternative, ultrasound elastography---a sonographic technique for interrogating tissue strains rather than acoustic properties---has been proposed as a possible tool for clinical diagnosis of DTI \cite{gehin06,gefen09,gefen13}. Some exploratory studies have successfully used this technique to quantify deep tissue injury formation not only numerically, but in {PVA}-cryogel phantoms as well as in a rat model \cite{deprez07,deprez11}. While these studies show promise, they are only the beginning for the adoption of ultrasound elastography as a viable clinical detection modality for deep tissue injuries.

			Recently, another possible avenue for DTI detection has arisen which lies in the biochemical markers present in a patient's blood or urine. Rhabdomyolysis refers to the process when myoglobin proteins from damaged skeletal muscle enter the bloodstream due to a breakdown of muscle fibres in the body. Although this condition may be caused by numerous factors such as hyperthermia, ingestion of various drugs, alcohol abuse, toxins, autoimmune disease, or physical damage \cite{beetham00,sauret02}, it may also be an indicator of formative DTI in at-risk patients who do not present with any of the aforementioned risk factors. Myoglobin proteins present in the blood get filtered in the kidneys and as such can present in the urine, turning it tea-brown \cite{bagley07}.

			With the many avenues of DTI detection currently being explored and utilized, it is most likely that a combination of all the techniques will provide the most utility. For example, upon hospital admission or with a reasonably high risk assessment score, a patient may be given a blood test which confirms the presence of a forming injury or not. Patients with forming injuries may then be scanned using ultrasound technology to locate and quantify the injury. That patient may then receive more targeted care, of which the effectiveness may be continually monitored using both blood and ultrasound tests. It is expected that the targeted care that this approach would provide would increase patient health and well-being while at the same time decreasing the overall load on the health-care system.

	\section{Ultrasound Elastography}
		Ultrasound elastography is a relatively new imaging modality which is capable of imaging the stiffness of soft tissue using ultrasound waves \cite{greenleaf03} and has its roots in the millennials-old clinical practise of manually palpating tissues to detect localized changes in the mechanical properties of the tissue \cite{adams02}. In general, the principle of ultrasound elastography is to visualize the deformation of soft tissue in response to an externally applied force \cite{brusseau00}. This is in contrast to traditional ultrasound images which are created by interrogating tissue with high-frequency acoustic waves and ``listening'' to their echoes as they reflect off of tissue boundaries and small tissue irregularities (scattering centres) \cite{hoskins10}. The externally applied force in ultrasound elastography may come from manual indentation of the ultrasound probe, a secondary external vibrator, or as an acoustic radiation force impulse (ARFI) generated by the ultrasound transducer itself \cite{greenleaf03}. Ultrasound elastography is a proven technology when it comes to detecting very stiff lesions against relatively unstiff backgrounds---it has successfully been used to detect breast and prostate cancer lesions \cite{tanter08,konig05}, liver fibrosis \cite{sandrin03,karlas12}, and atherosclerosis \cite{maurice04}. There are generally three distinct methodologies or algorithms for generating soft tissue elastograms: quasi-static methods which rely upon the manual indentation of the transducer probe; ARFI imaging which measures the dynamic response of tissue due to ARFI excitation; and shear wave speed quantification which measures shear wave speeds developed in tissue due to ARFI excitation.

		\comment{
			papers:
				\cite{doyley12}: review of 3 types of elastography wrt modelling and & inversion
				\cite{pavan10}: phantom materials for elasticity imaging
				\cite{fromageau03}: phantom material (PVA cryogel)
				\cite{osanai11}: quantifying skin elasticity
		}

		\subsection{Quasi-Static Ultrasound Elastography}
			Quasi-static ultrasound elastography was the earliest and most simple form of ultrasound elastography \cite{dickinson82,wilson82} and generally operates by cross-correlating axial scan lines of tissue in pre- and post- deformed states. The term ``quasi-static'' is used in this method as the deformation applied to the tissue is very slow compared to the measurement time. Quasi-static ultrasound elastography provides a qualitative measure of stiffness as the mechanical conditions involved during quasi-static interrogation cannot be fully known. Despite this, it is possible to obtain relative stiffness estimates by comparing lesionous regions against background tissue with a high spatial resolution and without modification to conventional ultrasound hardware \cite{chen96,treece11}. While quasi-static elastography originally relied upon one-dimensional ultrasound A-lines, the technique has since advanced to two-dimensional B-mode images \cite{ophir91} and even three-dimensional B-mode images \cite{deprez07,deprez09}.

			The cross-correlation foundation of quasi-static ultrasound elastography works by tracking the displacement of scattering centres which are inherently anchored to the tissue they are embedded within \cite{meunier95,vargheese09} in much the same manner as contact free strain measurements may be obtained using optic means \cite{austrell}. There have been numerous different quasi-static strain estimation algorithms developed, each with various advantages and disadvantages \cite{treece11}. The most common algorithm involves simple cross-correlation maximization and was among the first algorithms to be proposed \cite{ophir91}. One of the most promising algorithms models compressed regions of interest as both scaled and translated versions of their uncompressed counterparts \cite{brusseau00,brusseau08} which can overcome poor correlations in simpler algorithms due to warping of the tissue under compression. This technique has successfully been used to investigate a deep tissue injury in: a finite-element model; a tissue phantom; and a rat model \cite{deprez11}.

			\comment{
				papers:
					\cite{rivaz11}: using three b-modes to do qs use
					\cite{khaled06}: basic qs-use for quantifying tissue properties
			}

		\subsection{Acoustic Radiation Force Impulse Imaging}
			Acoustic radiation force impulse imaging is a more recent alternative to quasi-static ultrasound elastography which may greatly increase the inter-operator reliability of the technique by precisely controlling the externally applied mechanical interrogation force \cite{nightingale00-1,nightingale01}. While quasi-static ultrasound elastography relies upon the ultrasound operator to manually indent the tissue, ARFI imaging generates spatially focused ultrasound waves for relatively long periods of time (10s to 100s of \si{\micro\second}) compared to typical diagnostic procedures in order to generate an acoustic radiation force at the focal region \cite{nightingale02,nightingale02-7,palmeri05}. Once the acoustic radiation force is generated within the tissue, the procedure is extremely similar to the technique used in quasi-static ultrasound elastography---the deformation in the tissue caused by the externally applied force is tracked using classical ultrasound beams at high sampling frequencies. Although the magnitude of the resulting deformation is generally less than \SI{20}{\um} \cite{nightingale02}, ultrasound beams are still able to detect deformations of less than \SI{2}{\um} \cite{SiemensVirtualTouch,pinton06}. By comparing the level of deformation throughout the tissue to a homogeneous interrogation force, the relative stiffness of individual regions of tissue may be ascertained---relatively stiff regions of tissue will deform less than relatively unstiff regions of tissue.

			Since the development of acoustic radiation force within deep tissue requires greater amounts of applied pressure for longer durations than classical ultrasound b-mode imaging, the safety of the technique becomes an important consideration. Health Canada guidelines assert that ultrasound technologies be applied to patients only when medically necessary and exposures should be kept as low as reasonably achievable in any imaging mode \cite{HealthCanadaUltrasound}. Health Canada also places a limit on the derated spatial-peak time-average intensity, $I_{SPTA}$ of \SI{720}{\mW\per\cm\squared} which is derived from FDA regulations \cite{FDAUltrasoundGuidelines}. Since $I_{SPTA}$ is a temporal average, the compliance of this value can be easily controlled by controlling the repetition time of scans---simply disabling ultrasonic push beams for long enough that the tissue has a chance to recover from the initial burst of pressure will allow the device to operate with safety in regards to this parameter. A much more critical parameter is the derated spatial-peak pulse-average intensity, $I_{SPPA}$, which is not considered by Health Canada but limited to \SI{190}{\W\per\cm\squared} by the FDA \cite{FDAUltrasoundGuidelines} for classical diagnostic imaging. This measure is important as it relates to the peak intensity developed in the tissue during an interrogation pulse and is a limit that may need to be pushed in order to develop adequate deformation in deep tissues. Later revisions to FDA guidelines have allowed greater values of $I_{SPPA}$ for alternate imaging modes, with values up to \SI{933}{\W\per\cm\squared} being allowed for combined B and M-mode imaging \cite{hoskins10}.

		\subsection{Shear Wave Speed Quantification}
			Shear wave speed quantification represents the most complicated method of interrogating tissue stiffness using ultrasound technology, however these added complications come with the ability to obtain quantitative measures of tissue stiffness instead of the qualitative measures presented by quasi-static elastography and ARFI imaging. Unlike the longitudinal waves that are used in classical ultrasound imaging, shear waves travel perpendicular to the direction of particle motion and travel at relatively low speeds of \SI{1}{\m\per\s} -- \SI{10}{\m\per\s}. The speed of travel of shear waves is highly dependent on the density and stiffness of tissue as per equation \ref{equ:litreview_shear_speed} where $\mu_{tissue}$ is the shear modulus of the tissue and $\rho$ is the density. Since density cannot be measured \emph{in vivo}, the measurement of shear speed and an assumption of tissue density can be used to calculate the shear modulus \cite{hoskins10}.

			\begin{equation}
			\label{equ:litreview_shear_speed}
				c_T = \sqrt{\frac{\mu_{tissue}}{\rho}}
			\end{equation}

			In order to generate shear waves in tissue, a focused acoustic radiation impulse force must be applied to the tissue to generate shear waves which radiate from the focal point, much like creating ripples in a pond or ringing a bell \cite{nightingale03}. As these shear waves travel outwards from the ARFI focal point, the deformation they create in the tissue can be tracked using classical ultrasound beams sampled at extremely high frequencies. In order to image a large region of tissue at once using generated shear waves, a series of progressively deepening acoustic radiation impulses may be applied in an axial line to generate a ``mach cone'' of shear waves throughout an entire region of tissue \cite{bercoff04}.

			Shear wave speed quantification has been successfully used to noninvasively determine the mechanical properties of not only tissue mimicking materials \cite{cao13} but numerous human soft tissues \emph{in vivo} \cite{arda11}. Further work has been done to construct various viscoelastic models of soft tissue behaviour based on shear wave speed elastography including Kelvin-Voigt, Maxwell, and Zener models \cite{chen12,amador12}. Even more complete models have been constructed by combining shear wave speed quantification with ultrasonic computed tomography to calculate not only the shear modulus of tissue but the bulk modulus as well \cite{glozman10}. Finally, shear wave speed elastography has successfully been used to investigate crush injuries in rabbits which are aetiologically similar to deep tissue injuries \cite{lv21}.

			\comment {
				papers:
					%\cite{bercoff04}: generating supersonic shear wave mach cones to quantify elasticity across an entire region
					%\cite{amador12}: quantify viscoelastic material properties by estimating the complex shear elastic modulus
					%\cite{chen12}: Comparing soft tissue models (Voight, Maxwell, and Zener) to fit shear velocity measurements -> shear moduli
					%\cite{cao13}: using shear imaging to quantify elastic properties of tissue mimicking materials
					%\cite{arda11}: measure elasticity of various tissues using shear wave speed
					%\cite{glozman10}: combining shear wave + longitudinal wave speed to calculate bulk modulus of various tissues
					%\cite{lv21}: USE examining crush injury in rabbits
					%\cite{nightingale03}: generating shear waves using ARFI
			}

	\section{Conclusion}
		Pressure ulcers and deep tissue injuries are severe wounds that place a tremendous burden not only on those who suffer from them, but on the health care system as well. These injuries are generally caused by some combination of ischemia and reperfusion injury as well as excessive cell deformation. Deep tissue injuries are substantially more difficult to detect than pressure ulcers due to where they form---DTI generally form deep in tissue immediately superior to boney prominences and follow a ``bottom-to-top'' tunnelling pattern that is hardly detectable until it is ``too late'' and the wound has broken open as a late-stage pressure ulcer. Deep tissue injury prevention generally relies upon mechanically offloading at-risk tissue areas by ``turning'' the patient or by utilizing special pressure-redistribution support surfaces, while deep tissue injury treatment is somewhat limited and relies upon increasing a patient's overall health or resorting to surgical techniques. Recent research suggests that intermittent electrical stimulation may provide substantial benefits for preventing deep tissue injuries, however without a feasible means of reliably detecting them, the effectiveness of these treatments cannot be adequately gauged.

		While detection of deep tissue injuries may be done in a research setting by using $\mathrm{T}_2^*$-weighted MRI, this is not a cost-effective approach and is generally not used clinically. Instead, health-care practitioners rely upon risk-assessment scales which provide a somewhat subjective and qualitative measure of a patient's chance of forming a pressure ulcer or DTI instead of actually detecting the disease. Relatively recent advances in ultrasound technology may be able to bridge this gap by imaging the relative stiffness of tissue since it is known that deep tissue injuries undergo significant stiffness changes through their lifetime. The technique of using ultrasound to image tissue stiffness is called ``ultrasound elastography'' and it works through the estimation of relative local tissue deformations under a commonly applied load. Ultrasound elastography generally encompasses three main techniques of interrogating tissue: manually by indenting the transducer head and tracking displacement of scattering centres before and after the deformation; utilizing an acoustic radiation force to specifically displace a region of tissue and measuring it's dynamic response; and utilizing an acoustic radiation force to generate shear waves in the tissue and measuring the shear wave speeds as they travel through the tissue.

		While preliminary work has shown that quasi-static ultrasound elastography has the potential to be used for the early detection of deep tissue injuries \cite{deprez11}, the technique is not yet fully understood in this regard. Further, the use of ARFI imaging and shear wave speed quantification have not yet been explored as a means of detecting DTI. In order to advance the science and move closer to a clinical implementation of this technology, all modes of ultrasound elastography must be characterized with regards to their use in detecting deep tissue injuries.

\comment{
	\cleardoublepage

	\phantomsection

	\addcontentsline{toc}{section}{References}
	\bibcomplete{references}
	\printbibliography[heading=subbibliography]
}