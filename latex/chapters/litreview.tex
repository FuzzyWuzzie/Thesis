\chapter{Literature Review}
\label{chap:litreview}
	\section{Introduction}
		\note[KH]{Should this maybe go in the introduction chapter? If so, what should go here?}Pressure ulcers are an extraordinarily large problem facing the health care system today. Not only are billions of dollars spent annually treating these injuries but they also place an extremely significant burden on the people who suffer from them. Deep tissue injuries are currently considered a subset of pressure ulcers, although it is hypothesized that due to a lack of clinical detection modalities, most deep tissue injuries go undiagnosed until they characteristically ``break through'' the surface of the skin, at which point they are diagnosed as stage III or IV ulcers instead of an advanced deep tissue injury. While deep tissue injury prevention and mitigation strategies exist, the lack of a substantial method of detecting and diagnosing these injuries leads to an overall lack of coordination and sub-par treatment outcomes. The inclusion of a viable means of detecting early deep tissue injuries can only serve to strengthen the prevention and mitigation strategies that exist and give not only health-care practitioners but also researchers a solid foundation to base their treatments off of.

	\section{Deep Tissue Injuries}
		Deep tissue injuries are regions of subcutaneous tissue breakdown characterized by severe necrosis of the tissue. These injuries most commonly form immediately superior to boney prominences and gradually ``tunnel'' towards the surface of the skin. Upon superficial rupture, the wound is characterized by ``full thickness tissue loss'' with slough or eschar potentially present.

		According to the National Pressure Ulcer Advisory staging system, deep tissue injuries are defined as a ``purple or maroon localized area of discolored intact skin or blood-filled blister due to damage of underlying soft tissue from pressure and / or shear'' \cite{npuap07}. The definition goes on to say that ``the area may be preceded by tissue that is painful, firm, mushy, boggy, warmer, or cooler as compared to to adjacent tissue.'' This definition is somewhat vague, however, and the exact aetiology of this injuries will be discussed in detail in Section \ref{sec:litreview-aetiology}. Gunningberg et al. suggest that in order to reduce the prevalence of pressure ulcers, skin and risk assessment should be done as early as possible \cite{gunningberg08}.

		\begin{itemize}
			\item TALK ABOUT PREVALENCE AND COST HERE
			\item Hospital-acquired pressure ulcers: a comparison of costs in medical vs. surgical patients \cite{beckrich99}
			\item Pressure ulcer research funding in America: creation and analysis of an on-line database \cite{zanca03}
			\item Pressure ulcers in America: prevalence, incidence, and implications for the future. An executive summary of the National Pressure Ulcer Advisory Panel monograph \cite{npuapexecutive}
			\item Hospitalzations related to pressure ulcers among adults 18 years and older, 2006: statistical brief no. 64 \cite{russo08}
			\item Pressure ulcers in the elderly: analysis of prevalence and risk factors \cite{freitas11}
			\item Assessment and management of pressure ulcers in the elderly: current strategies \cite{jaul10}
			\item Pressure ulcers: the great insult \cite{maklebust05}
			\item Reducing pressure ulcer prevalence rates in the long-term acute care setting \cite{milne09}
			\item In-hopsital medical complications, length of stay, and mortality among stroke unit patients \cite{ingeman11}
			\item Tracking quality over time: what do pressure ulcer data show? \cite{gunningberg08}
			\item Recurrence of initial pressure ulcer in persons with spinal cord injuries \cite{niazi97}
			\item Pressure ulcers can form from being on the operating table \cite{aronovitch99}
		\end{itemize}

		\subsection{Aetiology and Histology}
			\label{sec:litreview-aetiology}
			Deep tissue injuries are thought to occur through the combinatory effects of three distinct but related mechanisms: ischemia, insufficient lymph drainage, and cell deformation. Ischemia is a condition where the blood supply to tissue has been cut off, rendering the tissue unable to function appropriately. Ordinarily, anaerobic metabolic waste products are carried from tissue via the lymph system, however if the lymph vessels become occluded, the waste products may accumulate in the tissue and cause harm. Cell deformation occurs when mechanical strains are imparted upon the tissue, causing excessive deformation in not only the extracellular matrix, but in the cells as well. Taken together, the presence of these two factors has been shown to greatly increase the risk of developing a deep tissue injury \cite{stekelenburg08}.

			For quite some time, ischemia was regarded as the chief acute risk factor for developing late-stage pressure ulcers \cite{witkowski82,dinsdale74,kosiak61}. Although studies have shown that healthy tissue is able to survive complete ischemia for approximately 4 hours before severe necrosis set in \cite{labbe87,strock69}, deep tissue injuries are clinically found when loading times are substantially less than this \cite{aronovitch99,bliss99}. The model of ischemic damage alone could not account for the rate of late-stage pressure ulcers that we were witnessed.

			Once it was realized that ischemia alone could not be the culprit behind deep tissue injury formation, ischemia-induced reperfusion injury became implicated in the formation of DTI \cite{Ytrehus95,Blaisdell02,tsuji05}. An ischemia-induced reperfusion injury is caused when blood is allowed to flow back into a region of tissue that was previously ischemic. While seeming somewhat contrary to its expected effect, the restoration of circulation results in a swelling and inflammatory effect which causes great microvascular damage \cite{Blaisdell02}. The effect of reperfusion was confirmed when comparing pure ischemic conditions in tissue to a cycle of ischemic-reperfused conditions over the same period of time, where it was found that significantly greater damage was caused by repeated loading-unloading rather than simple constant loading \cite{tsuji05}. While ischemia-reperfusion injuries provide a more complete explanation about the formation of deep tissue injuries, they still do not account for those injuries acquired under constant pressure over short time periods.

			Lymph drainage \cite{krouskop78,miller81,reddy81,braden87} --- where to fit this in?

			In order to account for deep tissue injuries that form over short time periods, a model of cell deformation leading to necrosis has more recently been proposed \cite{landsman95,bouten01,wang05}. It has constantly been observed that tissue regions which eventually form deep tissue injuries exhibit signs of locally increased strains \cite{stekelenburg06,ceelen08,linderganz08,portnoy09}, with greater degrees of deformation correlating to greater degrees of damage. To account for these results, it has been proposed that excessively deforming strains applied to cells over extended periods of time can alter the permeability of the cell's plasma membranes, leading to an overall reduced cell viability \cite{slomka12}. 

			There have been many models of deep tissue injury formation throughout the years, each relating to different mechanisms, though all relating to mechanical stress of the tissue, either through vessel occlusion or direct cellular strain. The truth is most likely a combination of these effects, with cell deformation dominating the damage on shorter time scales with increased applied pressure and vessel occlusion type injuries dominating on longer time scales \cite{stekelenburg08}.

			\begin{itemize}
				\item Compression induced damage and internal tissue strains are related \cite{ceelen08}. With a combined animal-experimental numerical approach, we show that there is a reproducible monotonic increase in damage with increasing maximum shear strain once a strain threshold has been exceeded.
				\item Computational model that shows how cells that die under compression decrease in stiffness. \cite{ceelen08-8}
				\item How does muscle stiffness affect the internal deformatins within the soft tissue layers of the buttocks under constant loading? \cite{loerakker13}
				\item A theoretical model to study the effects of cellular stiffening on the damage eveolution in deep tissue injury \cite{nagel09}
				\item The biomechanics of sitting-acquired pressure ulcers in patients with spinal cord injury or lesions \cite{gefen07-9}
				\item Assessment of mechanical conditions in sub-dermal tissues during sitting: a combined expeirmental-MRI and finite-element approach \cite{linderganz06}
				\item Mechanical gluteal soft tissue material parameter validation under complex tissue loading \cite{then09}
				\item Diffusion of water in skeletal muscle tissue is not influenced by compression in a rat model of deep tissue injury \cite{vanNierop10}.
				\item Pressure ulcer risk factors among hospitalized patients with activity limitation \cite{allman95}
				\item The etiology of pressure ulcers: skin deep or muscle bound? \cite{bouten03}
				\item Histopathology of pressure ulcers as a result of sequential computer-controller pressure sessions in a fuzzy rat model \cite{salcido94}
				\item National Pressure Ulcer Advisory Panel's Updated Pressure Ulcer Staging System \cite{black07}
				\item Diffusion of ulcers in the diabetic foot is promoted by stiffening of plantar muscular tissue under excessive bone compression \cite{gefen04}
				\item Risk factors for a pressure-related deep tissue injury: a theoretical mode \cite{gefen07}
				\item Deep tissue injury: how deep is our understanding? \cite{stekelenburg08}
				\item Deep tissue injury from a bioengineering point of view \cite{gefen09}
				\item Distribution of internal pressure around bony prominences: implications to deep tissue injury and effectiveness of intermittent electrical stimulation \cite{solis12-02}
				\item Distribution of internal strains around bony prominences in pigs \cite{solis12-03}
				\item The importance of internal strain as opposed to interface pressure in the prevention of pressure related deep tissue injury \cite{oomens10}
				\item Stress analyses coupled with damage laws to determine biomechanical risk factors for deep tissue injury during sitting \cite{linderganz09}
				\item Decubitus ulcers: role of pressure and friction in causation \cite{dinsdale74}
				\item Etiologic factors in pressure sores: an experimental model \cite{daniel81}
				\item In vivo muscle stiffening under bone compression promotes deep pressure sore \cite{gefen05}
				\item Pressure ulcers: avoidable or unavoidable? Results of the National Pressure Ulcer Advisory Panel Consensus Conference \cite{black11}
				\item The decubitus ulcer: facts and controversies \cite{campbell10}
				\item Mechanical compression-indeced pressure sores in rat hindlimb: muscle stiffness, histology, and computational models \cite{linderganz04}
			\end{itemize}

		\subsection{Detection}
			\begin{itemize}
				\item Combination of thermographic and ultrasonographic assessments for early detection of deep tissue injury \cite{higashino12}
				\item A review of deep tissue injury development, detection, and prevention \cite{gefen13}
				\item Pressure ulcer knowledge in medical residents: an opportunity for improvement \cite{levine12}
				\item Clinical nurse' knowledge and visual differentiation ability in pressure ulcer classification system and incontinence-associated dermatitis \cite{lee13}
				\item Toward real-time detection of deep tissue injury risk in wheelchair users using Hertz contact theory \cite{agam08}
				\item Compression-induced deep tissue injury examined with magnetic resonance imaging and histology \cite{stekelenburg06}
				\item Low-echoic lesions underneath the skin in subjects with spinal-cord injury \cite{kanno09}
				\item National Pressure Ulcer Advisory Panel's Updated Pressure Ulcer Staging System \cite{black07}
				\item A new pressure ulcer risk assessment scale for individuals with spinal cord injury \cite{salzberg96}
				\item Deep tissue engineering from a bioengineering point of view \cite{gefen09}
				\item Evaluation of four non-invasive methods for examination and characterization of pressure ulcers \cite{andersen08}
				\item Which techniques to improve the early detection and prevention of pressure ulcers? \cite{gehin06}
				\item Inception and validation of a pressure ulcer risk scale in oncology \cite{fromantin11}
				\item Assessment and management of pressure ulcers in the elderly: current strategies \cite{jaul10}
				\item Reliability testing of the national database of nursing quality indicators pressure ulcer indicator \cite{hart10}
				\item Ultrasound assessment of deep tissue injury in pressure ulcers: possible prediction of pressure ulcer progression \cite{aoi08}
				\item 3D ultrasound elastography for early detection of lesions. Evaluation on a pressure ulcer mimicking phantom \cite{deprez07}
			\end{itemize}

		\subsection{Prevention and Treatment}
			\begin{itemize}
				\item Prevention of deep tissue injury through muscle contractions induced by intermittent electrical stimulation after spinal cord injury in pigs \cite{solis13}
				\item The importance of internal strain as opposed to interface pressure in the prevention of pressure related deep tissue injury \cite{oomens10}
				\item New methodology for preventing pressure ulcers using actimetry and autonomous nervous system recording \cite{meffre06}
				\item Pressure ulcers: the great insult \cite{maklebust05}
				\item Reaching for the moon: achieving zero pressure ulcer prevalence, an update \cite{bales11}
				\item Distribution of Internal Pressure around Bony Prominences: Implications to deep tissue injury and effectiveness of intermittent electrical stimulation \cite{solis12-02}
				\item Distribution of internal strains around bony prominences in pigs \cite{solis12-03}
				\item Reaching for the moon: achieving zero pressure ulcer prevalence, an update. \cite{bales11}
				\item Reducing pressure ulcers in hip fracture patients \cite{thompson11}
				\item Reducing pressure ulcer prevalence rates in the long-term acute care setting \cite{milne09}
				\item Development of pressure ulcer program across a university health system \cite{carson11}
				\item Assessment and management of pressure ulcers in the elderly: current strategies \cite{jaul10}
				\item Pressure ulcers: the great insult \cite{maklebust05}
				\item Intermittent electrical stimulation redistributes pressure and promotes tissue oxygenation in loaded muscles of individuals with spinal cord injury. \cite{gyawali11}
			\end{itemize}

	\section{Ultrasound Elastography}
		\lipsum[1]

		\subsection{Quasi-Static Ultrasound Elastography}
			\begin{itemize}
				\item A new method for the visualization and quantification of internal skin elasticity by ultrasound imaging \cite{osanai11}
			\end{itemize}

		\subsection{Acoustic Radiation Force Impulse Imaging}
			\begin{itemize}
				\item derp
			\end{itemize}

		\subsection{Shear Wave Speed Quantification}
			\begin{itemize}
				\item derp
			\end{itemize}

	\section{Numerical Characterization / Finite-Element Modelling}
		\lipsum[1]

	\section{Conclusion}
		\lipsum[1]

	\cleardoublepage

	\phantomsection

	\addcontentsline{toc}{section}{References}
	%\bibliographystyle{IEEEtran}
	%\renewcommand{\bibliography}[1]{}
	%\bibliography{references}
	\bibcomplete{references}
	\printbibliography[heading=subbibliography]