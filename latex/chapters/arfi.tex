\chapter{Numerical Characterization of Acoustic Radiation Force Impulse Imaging}
	\label{chap:arfi}
	\section{Introduction}
		Acoustic radiation force impulse imaging presents a chief benefit over quasi-static ultrasound elastography in that since the external deformation force is applied by the transducer itself

	\section{Methods}
		%\lipsum[1]
		\subsection{Numerical Model}
			%\lipsum[1]
			\subsubsection{Governing Equations}
				The governing equations use for this model were the set of coupled first-order partial differential equations \ref{equ:arfi_gov}. These equations are the first-order equivalents of \ref{equ:wave_equation} taking into account acoustic absorption, tissue heterogeneities, and acoustic wave non-linearities \cite{treeby12}.

				\begin{subequations}
					\label{equ:arfi_gov}
					\begin{align}
						\frac{\partial \mathbf{u}}{\partial t} &= - \frac{1}{\rho_0} p \label{equ:arfi_gov_p1} \\
						\frac{\partial p}{\partial t} &= -\left(2 \rho + \rho_0\right)\nabla \cdot \mathbf{u} - \mathbf{u} \cdot \nabla \rho_0 \label{equ:arfi_gov_p2} \\
						p &= c_0^2 \left(\rho + \mathbf{d} \cdot \nabla \rho_0 + \frac{B}{2A} \frac{\rho^2}{\rho_0} - L\rho \right) \label{equ:arfi_gov_p3}
					\end{align}
				\end{subequations}

				\begin{equation}
					\label{equ:wave_equation}
					\nabla^2 p - \frac{1}{c_0^2}\frac{\partial^2 p}{\partial t^2} = 0
				\end{equation}

			\subsubsection{Boundary and Initial Conditions}
				%\lipsum[1]
	\section{Results}
		\begin{figure}[!t]
			\centering
			\subfloat[]{
				\begin{tikzpicture}
					\begin{axis}[
						scale only axis,
						enlargelimits=false,
						unit vector ratio*=1 1 1,
						height=2in,
						y dir=reverse,
						xlabel={X-Coordinate, $x$ (\si{\cm})},
						ylabel={Depth, $d$ (\si{\cm})},
						axis on top,]
							\addplot graphics[xmin=-2,xmax=2,ymin=0,ymax=10]{assets/shear/data/Fx158.png};
					\end{axis}
				\end{tikzpicture}
				\label{fig:shear_arfi_fx}
			}
			\subfloat[]{
				\begin{tikzpicture}
					\begin{axis}[
						scale only axis,
						enlargelimits=false,
						unit vector ratio*=1 1 1,
						height=2in,
						y dir=reverse,
						xlabel={X-Coordinate, $x$ (\si{\cm})},
						axis on top,]
							\addplot graphics[xmin=-2,xmax=2,ymin=0,ymax=10]{assets/shear/data/Fy158.png};
					\end{axis}
				\end{tikzpicture}
				\label{fig:shear_arfi_fy}
			}
			\subfloat[]{
				\begin{tikzpicture}
					\begin{axis}[
						scale only axis,
						enlargelimits=false,
						unit vector ratio*=1 1 1,
						height=2in,
						y dir=reverse,
						xlabel={X-Coordinate, $x$ (\si{\cm})},
						axis on top,
						colormap/jet, colorbar, point meta min=-27.5, point meta max=27.5, colorbar style={at={(1.05,0)}, anchor=south west, width=0.03\textwidth, ylabel={Acoustic Radiation Force, $F_{ARFI}$ (\si{\kN\per\metre\cubed})},
						draw=black, text=black, fill=black}]
							\addplot graphics[xmin=-2,xmax=2,ymin=0,ymax=10]{assets/shear/data/F158.png};
					\end{axis}
				\end{tikzpicture}
				\label{fig:shear_arfi_fsum}
			}
			\caption[]{}
			\label{fig:shear_arfi_forces}
		\end{figure}

		\begin{figure}[!t]
			\centering
			\begin{tikzpicture}
				\begin{axis}[
					scale only axis,
					height=3in,
					width=\textwidth-\widthof{100}-1in,
					xlabel={Focal Depth, $d_{f}$ (\si{\cm})},
					ylabel={Body Force at Focal Point, $F_{b,f}$ (\si{\newton\per\metre\cubed})},
					grid=major,
					legend entries={$f = \SI{1}{\MHz}$, $f = \SI{2}{\MHz}$, $f = \SI{4}{\MHz}$, $f = \SI{6}{\MHz}$},
					legend style={legend pos=north east,font=\small},
					clip=true,
					cycle list name=ColourPlotCycle,
					draw=black, text=black, fill=black]
					\addplot table {assets/arfi/data/depth_force_freq1.dat};
					\addplot table {assets/arfi/data/depth_force_freq2.dat};
					\addplot table {assets/arfi/data/depth_force_freq4.dat};
					\addplot table {assets/arfi/data/depth_force_freq6.dat};
				\end{axis}
			\end{tikzpicture}
			\caption[Lessening of ARFI with increasing depth and probing frequency]{Lessening of ARFI with increasing depth and probing frequency}
			\label{fig:freq-depth-force}
		\end{figure}

		\begin{figure}[!t]
			\centering
			\begin{tikzpicture}
				\begin{axis}[
					scale only axis,
					height=3in,
					width=\textwidth-\widthof{100}-1in,
					xlabel={Focal Depth, $d_{f}$ (\si{\cm})},
					ylabel={Spatial-Peak Pulse-Average Intensity, $I_{SPPA.3}$ (\si{\watt\per\cm\squared})},
					grid=major,
					legend entries={$f = \SI{1}{\MHz}$, $f = \SI{2}{\MHz}$, $f = \SI{4}{\MHz}$, $f = \SI{6}{\MHz}$, FDA-approved limit},
					legend style={legend pos=north east,font=\small},
					clip=true,
					cycle list name=ColourPlotCycle,
					draw=black, text=black, fill=black]
					\addplot table {assets/arfi/data/depth_isppa_freq1.dat};
					\addplot table {assets/arfi/data/depth_isppa_freq2.dat};
					\addplot table {assets/arfi/data/depth_isppa_freq4.dat};
					\addplot table {assets/arfi/data/depth_isppa_freq6.dat};
					\addplot[mark=none,dashed,ultra thick] plot coordinates {(1, 190) (9, 190)};
				\end{axis}
			\end{tikzpicture}
			\caption[$I_{SPPA.3}$ safety measures of ARFI pulses]{$I_{SPPA.3}$ safety measures of ARFI pulses}
			\label{fig:freq-depth-isppa}
		\end{figure}

		\begin{figure}[!t]
			\centering
			\begin{tikzpicture}
				\begin{axis}[
					scale only axis,
					height=3in,
					width=\textwidth-\widthof{100}-1in,
					xlabel={Focal Depth, $d_{f}$ (\si{\cm})},
					ylabel={Maximum Induced Tissue Displacement, $v_{max}$ (\si{\micro\metre})},
					grid=major,
					legend entries={$f = \SI{1}{\MHz}$, $f = \SI{2}{\MHz}$, $f = \SI{4}{\MHz}$, $f = \SI{6}{\MHz}$, Detection Threshold},
					legend style={legend pos=north east,font=\small},
					clip=true,
					cycle list name=ColourPlotCycle,
					draw=black, text=black, fill=black]
					\addplot table {assets/arfi/data/depth_maxDisp_freq1.dat};
					\addplot table {assets/arfi/data/depth_maxDisp_freq2.dat};
					\addplot table {assets/arfi/data/depth_maxDisp_freq4.dat};
					\addplot table {assets/arfi/data/depth_maxDisp_freq6.dat};
					\addplot[mark=none,dashed,ultra thick] plot coordinates {(1, 0.5) (9, 0.5)};
				\end{axis}
			\end{tikzpicture}
			\caption[Maximum tissue displacement generated by ARFI forces]{Maximum tissue displacement generated by ARFI forces}
			\label{fig:freq-depth-maxDisp}
		\end{figure}

		\begin{figure}[!t]
			\centering
			\begin{tikzpicture}
				\begin{axis}[
					scale only axis,
					height=3in,
					width=\textwidth-\widthof{100}-1in,
					xlabel={ARFI Frequency, $f_{ARFI}$ (\si{\MHz})},
					ylabel={Body Force at Focal Point, $F_{b,f}$ (\si{\newton\per\metre\cubed})},
					grid=major,
					legend entries={$w_{trans} = \SI{4}{\cm}$, $w_{trans} = \SI{8}{\cm}$, $w_{trans} = \SI{10}{\cm}$},
					legend style={legend pos=north east,font=\small},
					clip=true,
					cycle list name=ColourPlotCycle,
					draw=black, text=black, fill=black]
					\addplot table {assets/arfi/data/freq_force_width4.dat};
					\addplot table {assets/arfi/data/freq_force_width8.dat};
					\addplot table {assets/arfi/data/freq_force_width10.dat};
				\end{axis}
			\end{tikzpicture}
			\caption[Lack of transducer width effect on focal force]{Lack of transducer width effect on focal force}
			\label{fig:trans-width-force}
		\end{figure}

		\begin{figure}[!t]
			\centering
			\begin{tikzpicture}
				\begin{axis}[
					scale only axis,
					height=3in,
					width=\textwidth-\widthof{100}-1in,
					xlabel={Pulse Cycles, $n_c$},
					ylabel={Body Force at Focal Point, $F_{b,f}$ (\si{\newton\per\metre\cubed})},
					grid=major,
					clip=true,
					cycle list name=ColourPlotCycle,
					draw=black, text=black, fill=black,
					ymin=0, ymax=2e5]
					\addplot table {assets/arfi/data/pulse_cycles.dat};
				\end{axis}
			\end{tikzpicture}
			\caption[Lack of effect of pulse cycles on force at focal point]{Lack of effect of pulse cycles on force at focal point}
			\label{fig:pulse_cycles_force}
		\end{figure}

		\begin{figure}[!t]
			\centering
			\begin{tikzpicture}
				\begin{axis}[
					scale only axis,
					height=3in,
					width=\textwidth-\widthof{100}-1in,
					xlabel={Focal Depth, $d_f$ (\si{\cm})},
					ylabel={Body Force at Focal Point, $F_{b,f}$ (\si{\kN\per\metre\cubed})},
					grid=major,
					legend entries={$P_{source} = \SI{4}{\MPa}$, $P_{source} = \SI{5}{\MPa}$, $P_{source} = \SI{6}{\MPa}$, $P_{source} = \SI{7}{\MPa}$, $P_{source} = \SI{8}{\MPa}$},
					legend style={legend pos=north east,font=\small},
					clip=true,
					cycle list name=ColourPlotCycle,
					draw=black, text=black, fill=black]
					\addplot table[x expr=\thisrow{depth}*100, y expr=\thisrow{force}*1e-3] {assets/arfi/data/focal_force_depth_p4.dat};
					\addplot table[x expr=\thisrow{depth}*100, y expr=\thisrow{force}*1e-3] {assets/arfi/data/focal_force_depth_p5.dat};
					\addplot table[x expr=\thisrow{depth}*100, y expr=\thisrow{force}*1e-3] {assets/arfi/data/focal_force_depth_p6.dat};
					\addplot table[x expr=\thisrow{depth}*100, y expr=\thisrow{force}*1e-3] {assets/arfi/data/focal_force_depth_p7.dat};
					\addplot table[x expr=\thisrow{depth}*100, y expr=\thisrow{force}*1e-3] {assets/arfi/data/focal_force_depth_p8.dat};
				\end{axis}
			\end{tikzpicture}
			\caption[Strong dependence on source pressure of focal point force]{Strong dependence on source pressure of focal point force}
			\label{fig:pressure_force}
		\end{figure}

		\begin{figure}[!t]
			\centering
			\begin{tikzpicture}
				\begin{axis}[
					scale only axis,
					height=3in,
					width=\textwidth-\widthof{100}-1in,
					xlabel={Focal Depth, $d_f$ (\si{\cm})},
					ylabel={Maximum Induced Tissue Displacement, $\left|v_{max}\right|$ (\si{\micro\metre})},
					grid=major,
					legend entries={$P_{source} = \SI{4}{\MPa}$, $P_{source} = \SI{5}{\MPa}$, $P_{source} = \SI{6}{\MPa}$, $P_{source} = \SI{7}{\MPa}$, $P_{source} = \SI{8}{\MPa}$},
					legend style={legend pos=north east,font=\small},
					clip=true,
					cycle list name=ColourPlotCycle,
					draw=black, text=black, fill=black]
					\addplot table[x expr=\thisrow{depth}*100, y expr=\thisrow{maxDisp}*1e6] {assets/arfi/data/maxDisp_depth_p4.dat};
					\addplot table[x expr=\thisrow{depth}*100, y expr=\thisrow{maxDisp}*1e6] {assets/arfi/data/maxDisp_depth_p5.dat};
					\addplot table[x expr=\thisrow{depth}*100, y expr=\thisrow{maxDisp}*1e6] {assets/arfi/data/maxDisp_depth_p6.dat};
					\addplot table[x expr=\thisrow{depth}*100, y expr=\thisrow{maxDisp}*1e6] {assets/arfi/data/maxDisp_depth_p7.dat};
					\addplot table[x expr=\thisrow{depth}*100, y expr=\thisrow{maxDisp}*1e6] {assets/arfi/data/maxDisp_depth_p8.dat};
				\end{axis}
			\end{tikzpicture}
			\caption[]{}
			\label{fig:pressure_maxDisp}
		\end{figure}

		\begin{figure}[!t]
			\centering
			\begin{tikzpicture}
				\begin{axis}[
					scale only axis,
					height=3in,
					width=\textwidth-\widthof{100}-1in,
					xlabel={Probing Frequency, $f$ (\si{\MHz})},
					ylabel={Maximum Induced Tissue Displacement, $\left|v_{max}\right|$ (\si{\micro\metre})},
					grid=major,
					legend entries={$P_{source} = \SI{4}{\MPa}$, $P_{source} = \SI{6}{\MPa}$, $P_{source} = \SI{8}{\MPa}$},
					legend style={legend pos=north east,font=\small},
					clip=true,
					cycle list name=ColourPlotCycle,
					draw=black, text=black, fill=black]
					\addplot table[x expr=\thisrow{frequency}*100, y expr=\thisrow{maxDisp}*1e6] {assets/arfi/data/freq_maxDisp_p4.dat};
					\addplot table[x expr=\thisrow{frequency}*100, y expr=\thisrow{maxDisp}*1e6] {assets/arfi/data/freq_maxDisp_p6.dat};
					\addplot table[x expr=\thisrow{frequency}*100, y expr=\thisrow{maxDisp}*1e6] {assets/arfi/data/freq_maxDisp_p8.dat};
				\end{axis}
			\end{tikzpicture}
			\caption[]{}
			\label{fig:freq_pressure_maxDisp}
		\end{figure}