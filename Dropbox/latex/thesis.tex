\documentclass[12pt]{book}

\usepackage[top=1in, bottom=1in, left=1.5in, right=1.5in, letterpaper]{geometry}
\usepackage[hidelinks]{hyperref}
\usepackage{SIunits}
\usepackage{booktabs}
\usepackage{setspace}
\usepackage{trackchanges}
\usepackage{fancyhdr}
\usepackage{etoolbox}
\usepackage{lipsum}

% remove empty pages after \chapter commands
\let\cleardoublepage\clearpage

% set up our headers
\setlength{\headheight}{15.2pt}
\pagestyle{fancy}
\fancyhf{}
\renewcommand{\headrulewidth}{0pt}

\makeatletter
\renewcommand\chapter{\if@openright\cleardoublepage\else\clearpage\fi
                    \thispagestyle{fancyplain}% original style: plain
                    \global\@topnum\z@
                    \@afterindentfalse
                    \secdef\@chapter\@schapter}
\makeatother

\begin{document}

%% TITLE PAGE %%
	\begin{center}

	{\fontsize{14pt}{1em}\selectfont \textbf{University of Alberta}}
	\vspace{3em}

	{\fontsize{13pt}{1em}\selectfont Numerical Characterization of Ultrasound Elastography for the Early Detection of Deep Tissue Injuries}
	\vspace{2em}

	{\fontsize{10pt}{1em} by}
	\vspace{1em}

	{\fontsize{13pt}{1em} Kenton Hamaluik}
	\vspace{5em}

	{\fontsize{11pt}{1em} A thesis submitted to the Faculty of Graduate Studies and Research in partial fulfilment of the requirements for the degree of}
	\vspace{3em}

	{\fontsize{13pt}{1em} Master of Science}
	\vspace{4em}

	{\fontsize{13pt}{1em} Department of Mechanical Engineering}
	\vspace{5em}

	{\fontsize{11pt}{1em}\copyright\ Kenton Hamaluik \\
	Fall 2013 \\
	Edmonton, Alberta}
	\vfill

	{\fontsize{9pt}{1em} Permission is hereby granted to the University of Alberta Libraries to reproduce single copies of this thesis and to lend or sell such copies for private, scholarly or scientific research purposes only. Where the thesis is converted to, or otherwise made available in digital form, the University of Alberta will advise potential users of the thesis of these terms.
	\vspace{1em}

	The author reserves all other publication and other rights in association with the copyright in the thesis and, except as herein before provided, neither the thesis nor any substantial portion thereof may be printed or otherwise reproduced in any material form whatsoever without the author's prior written permission.}
	\end{center}

%% ABSTRACT %%
\chapter*{Abstract}
	\doublespacing

	\annote[Kenton]{Deep tissue injuries are subcutaneous regions of extreme tissue breakdown generally induced by the application of significant mechanical pressure over extended periods of time through the biological mechanisms of ischemia and cell deformation causing rupture}{Way too long}. These wounds are commonly suffered as a secondary wound or disease, often formed due to extended periods of motionless such as stationary sitting in spinal cord injured patients or those undergoing surgery. \note[KH]{Abstract should be $<=$ 150 words and give a decent summary of everything}

%% ACKNOWLEDGEMENT %%
\onehalfspacing
\chapter*{Acknowledgements}
	\lipsum[1]

%% TOC %%
\tableofcontents
\clearpage

\listoftables
\clearpage

\listoffigures
\clearpage

\setcounter{page}{1}
\doublespacing

%% INTRODUCTION %%
\rfoot[\thepage]{\thepage}
\chapter{Introduction}
	Ultrasound elastography is a relatively new ultrasonic imaging modality which utilizes traditional ultrasound waveforms to interrogate tissue stiffness rather than tissue echogenicity as is done in classic ultrasound imaging. The resulting tissue stiffness maps are referred to as elastograms, a terminology which will be used throughout this work. By examining displacement characteristics of tissue under load, the relative localized stiffness of the tissue may be ascertained. While regional tissue stiffness changes are to be expected due to the heterogeneous composition of generalized soft tissues, localized stiffness changes may be used as an indicator of tissue health \cite{gefen09} with relatively stiff tissues showing signs of rigor mortis and cell death and relatively soft tissues showing signs of tissue necrosis and decomposition. While ultrasound elastography has typically been used to investigate cancerous lesions this work seeks to use it as a means of detecting deep tissue injuries which as of the time of writing are not clinically detectable until they breach the surface of the skin.

	Before ultrasound elastography can be used clinically with any degree of certainty, the effect of numerous important parameters relating to the imaging modality must be understood and characterized. For example, chief parameters of interest include the depth of a lesion and its overall size---parameters which may immediately disqualify certain lesions from even being interrogated by diffused ultrasound beams and as a result would be invisible on the subsequent elastogram. Similarly, for the purposes of designing application-specific elastography transducers it becomes critical to fully understand the effect of transducer device parameters such as ultrasonic probing frequency and transducer f-number on the elastogram image quality and lesion contrast. In order to properly use ultrasound elastography to detect, diagnose, and monitor formative and progressive deep tissue injuries it is crucial to first fully understand and characterize the technology for this specific use.

	\section{Objective}
		The broad objective of this work was to numerically characterize the use of ultrasound elastography to detect and monitor formative and progressive deep tissue injuries. When the effect of numerous interrogation parameters is understood, the technology may be evaluated on its feasibility and usefulness to detect deep tissue injuries, with the ultimate goal that ultrasound elastography be implemented clinically for detecting deep tissue injuries.

	\section{Motivation}
		According to the National Pressure Ulcer Advisory Board, deep tissue injuries are classified as a sub-category of pressure ulcers \cite{black11}. Pressure ulcers and subsequently deep tissue injuries are commonly suffered by people with limited mobility, such as those undergoing lengthy surgical procedures, the elderly, and those with spinal cord injuries \cite{allman95} with up to \unit{80}{\%} of people with spinal cord injuries developing at least one pressure ulcer in their lifetime \cite{salzberg96}. While traditional pressure ulcers form in a ``top-to-bottom'' pattern [??], deep tissue injuries form in a ``bottom-to-top'' pattern, whereby the injury starts deep below the skin surface---often at the bone-muscle interface \cite{kanno09}. This nature of not being externally visible until the wound has severely progressed makes deep tissue injuries exceedingly difficult to not only diagnose but also to prevent and treat.

		As of the time of writing, there is no clinically feasible method of detecting deep tissue injuries until they begin to damage the skin---even the National Pressure Ulcer Advisory Panel's description of them is largely based on their appearance after the fact \cite{npuap07}. With our inability to detect these forming injuries and subsequently implement deep tissue injury prevention and mitigation protocols, the injuries may eventually progress to form large subcutaneous cavities which eventually break through the surface and reveal themselves as stage III or IV pressure ulcers \cite{bouten03,oomens10}. \note[KH]{Needs more}

	\section{Methodology}
		In order to investigate the use of ultrasound elastography for the detection of deep tissue injuries, the technology must first be characterized and fully understood. While traditional experimentation provides an opportunity to work with physical subjects it can be severely limiting as absolute control over all investigated parameters is relinquished. Further, subject recruitment may present an insurmountable barrier to the execution of such a study. As such, in this exploratory work, various numerical models of the technology have been utilized to investigate the controlled effect of a broad number of parameters relating to each technology. Specifically, finite-element models of ultrasonic wave propagation in heterogeneous soft tissues have been developed. \annote[KH]{These models were coupled with various tissue strain estimation algorithms and utilized to carry out parametric studies on the detection sensitivity of ultrasound elastography with respect to various lesion and technological parameters.}{Holy wordy batman!} Chief parameters of interest include those related to the physical realities of deep tissue injuries such as lesion depth, size, and relative mechanical stiffness as well as parameters related to the design and development of appropriate ultrasonic transducers such as probing frequency, transducer f-number, etc.

	\section{Thesis Outline}
		In this work, three methods of ultrasonic elastogram image formation have been investigated: quasi-static ultrasound elastography, acoustic radiation force impulse imaging, and shear wave speed quantification. While all three methods may be used to interrogate tissue stiffness utilizing the principles of ultrasound physics.

%% LITERATURE REVIEW %%
\chapter{Literature Review}
	\section{Introduction}
		\lipsum[1]
	\section{Deep Tissue Injuries}
		\lipsum[1]
		\subsection{Aetiology}
			\lipsum[1]
		\subsection{Treatment}
			\lipsum[1]
		\subsection{Detection}
			\lipsum[1]
	\section{Ultrasound Elastography}
		\lipsum[1]
		\subsection{Quasi-Static Ultrasound Elastography}
			\lipsum[1]
		\subsection{Acoustic Radiation Force Impulse Imaging}
			\lipsum[1]
		\subsection{Shear Wave Speed Quantification}
			\lipsum[1]
	\section{Numerical Characterization / Finite-Element Modelling}
		\lipsum[1]
	\section{Conclusion}
		\lipsum[1]

%% QUASI-STATIC %%
\chapter{Numerical Characterization of Quasi-Static Ultrasound Elastography}
	\section{Introduction}
		\lipsum[1]
	\section{Methods}
		\lipsum[1]
		\subsection{Finite-Element Model of Ultrasound Image Formation in Heterogeneous Soft Tissue}
			\lipsum[1]
			\subsubsection{Governing Equations}
				\lipsum[1]
			\subsubsection{Boundary and Initial Conditions}
				\lipsum[1]
		\subsection{Implementation of Tissue Strain Estimation Algorithm}
			\lipsum[1]
	\section{Results}
		\lipsum[1]
		\subsection{Lesion Depth Characterization}
			\lipsum[1]
		\subsection{Lesion Size Characterization}
			\lipsum[1]
		\subsection{Lesion Stiffness Characterization}
			\lipsum[1]

%% ARFI %%
\chapter{Numerical Characterization of Acoustic Radiation Force Impulse Imaging}
	\section{Introduction}
		\lipsum[1]
	\section{Methods}
		\lipsum[1]
		\subsection{Numerical Model}	\lipsum[1]
			\subsubsection{Governing Equations}
				The governing equations used for this model were the set of coupled first-order partial differential equations \ref{equ:arfi_gov_p1} -- \ref{equ:arfi_gov_p3}.
				\begin{equation}
					\label{equ:arfi_gov_p1}
					\frac{\partial \vec{u}}{\partial t} = - \frac{1}{\rho_0} p
				\end{equation}
				\begin{equation}
					\label{equ:arfi_gov_p2}
					\frac{\partial p}{\partial t} = - \rho_0 \nabla \cdot \vec{u}
				\end{equation}
				\begin{equation}
					\label{equ:arfi_gov_p3}
					p = c_0^2 p
				\end{equation}

			\subsubsection{Boundary and Initial Conditions}
				\lipsum[1]
	\section{Results}
		\lipsum[1]

%% SHEAR %%
\chapter{Numerical Characterization of Shear Wave Speed Quantification}
	\lipsum[1]

%% DISCUSSION %%
\chapter{Discussion}
	\lipsum[1]

%% CONCLUSION %%
\chapter{Conclusion}
	\lipsum[1]
	\section{Clinical Need for DTI Detection}
		\lipsum[1]
	\section{USE Provides Potential Diagnosis Capability}
		\lipsum[1]
	\section{Future Work}
		\lipsum[1]
		\subsection{Animal Studies?}
			\lipsum[1]
		\subsection{Human Studies?}
			\lipsum[1]

%% BIBLIOGRAPHY %%
\bibliographystyle{plain}
\bibliography{references}

%% APPENDICES %%
\appendix
\pretocmd{\chapter}{%
  \clearpage
  \pagenumbering{arabic}%
  \renewcommand*{\thepage}{\thechapter-\arabic{page}}%
}{}{}
\setcounter{page}{1}
\chapter{Source Code}
	\lipsum[1]
	\section{Quasi2DUltrasound}
		\lipsum[1]

\end{document}